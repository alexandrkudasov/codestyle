\documentclass{article}
\usepackage[utf8]{inputenc}
\usepackage[russian]{babel}
\usepackage{mathtext}
\title{code style}
\author{автор: Кудасов A.В., ИА-032,\\email: alex2002sib@gmail.com,\\ github: @alexandrkudasov}
\date{Февраль 2022}

\begin{document}

\maketitle

\section{Введение}
Основная цель рекомендаций — улучшение читаемости и, следовательно, ясности и лёгкости поддержки, а также общего качества кода например:
Как использовать отступы, когда и как ставить скобки всех видов, как оформлять комментарии и т.д.
\section{C++}

\subsection{Версия}
В настоящее время используется версия C++17 \cite{CPP}.
\subsection{Форматирование}
1. Большую часть форматирования сделает автоматически clang-format.\\
2. Отступы — 4 пробела. Настройте среду разработки так, чтобы таб добавлял четыре пробела.\\
3. Открывающая и закрывающие фигурные скобки на отдельной строке.\\
4. Вокруг бинарных операторов (+, -, *, /, \%, …) ставятся пробелы.\\\\
\begin{tabular}{ |l|}

\hline
 \\
inline void readBoolText(bool \& x, ReadBuffer \& buf)\\
\{\\
    char tmp = '0';\\
    readChar(tmp, buf);\\
    x = tmp != '0';\\
\}\\
\hline
\end{tabular}\\\\\\
5.Унарные операторы --, ++, *, &, … не отделяются от аргумента пробелом.\\
6. Для функций. Пробелы вокруг скобок не ставятся.\\
\begin{tabular}{ |l|}

\hline
 
inline void readBoolText(bool \& x, ReadBuffer \& buf)\\
\{\\
...\\
\}\\
\hline
\end{tabular}\\\\\\
7. В классах и структурах, public, private, protected пишется на том же уровне, что и class/struct, а остальной код с отступом.\\
\begin{tabular}{ |l|}

\hline
 template <typename T>\\
class MultiVersion\\
\{\\
public:\\
    using Version = std::sharedp<const T>;\\
    ...\\
\}\\
\hline
\end{tabular}\\\\\\
8. всегда ставьте фигурные скобки, даже если в операторе if нужно выполнить всего одну строку кода. Во вторых, переносите оператор else, на новую строку, хотя, на первый взгляд это может показаться и не удобно. В чем плюс? А плюс в том, что если вдруг вам нужно будет закомментировать или удалить блок кода с оператором else, то вам это будет сделать очень просто.
Вот запись, как делать не следует:\\
\begin{tabular}{ |l|}

\hline\\
 if (......)\\
\{\\
   ...\\
\}\\
else\\
\{\\
  ...\\
\}\\
\hline
\end{tabular}\\\\\\
\subsection{Названия функций и пременных}
Очень важно дать правильное название вашей фунуции или пременной, которое бы описовало их работу или содержимое.\\
\begin{tabular}{ |l|}

\hline
 \\
long double factorial(int N)\\
\{\\
    if(N < 0) \\
        return 0; \\
    if (N == 0) \\
        return 1; \\
    else \\
        return N * factorial(N - 1);\\
\}\\
\hline
\end{tabular}\\\\\\
Здесь по названию функции сразу понятно, что она делает.
\subsection{Рекомендации}
Используйте IDE для программирования.
 В мире существует очень много программ, которые подсвечивают, код, форматируют его и автоматически его собирают в готовую программу. Кроме всего этого, IDE информирует об возможных ошибках, еще до этапа сборки программы, подсвечивают неиспользуемые переменные, а также легко настраиваются под индивидуальные требования программистов. Все это реализовано с одной единственной целью — сделать процесс составления программ намного приятнее и намного быстрее.
 \begin{thebibliography}{3}
\bibitem{CPP}
 http://cppstudio.com/post/7873/
 \bibitem{СPP}
  https://habr.com/ru/company/ruvds/blog/574352/
\end{thebibliography}

\end{document}
